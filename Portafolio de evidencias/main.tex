\documentclass[letterpaper, 10 pt, conference]{article} 
\usepackage[spanish]{babel}
\usepackage{amsmath,amssymb,amscd,amsthm} % variety of useful math macros
\usepackage[inner=1.5 cm, outer = 1.5 cm, top=1 cm, bottom = 1.5 cm]{geometry}
\usepackage{subcaption}
%For inserting graphics
\usepackage{graphicx}
\usepackage[dvipsnames]{xcolor}
\usepackage{listings}
\usepackage[utf8]{inputenc}
\usepackage{hyperref}
\usepackage{array, multirow}
\usepackage{pdfpages}
\begin{document}

%\newpage
\thispagestyle{empty}
\vspace{10 cm}
\begin{scshape}
\begin{center}
	{$\,$} \\[20 mm]
	{\Large{Universidad Autónoma de Nuevo León}} \\[5mm]
	{\large{Facultad de Ingeniería Mecánica y Eléctrica}} \\[5mm]
	{\large{Posgrado en Ciencias de la Ingeniería con Orientación en Nanotecnología}} \\[5 mm]
	{\large{Maestría}}
	\vskip16mm
	\begin{figure}[h!]
		\centering
		\begin{subfigure}{0.3\linewidth}
			\includegraphics[width=\linewidth]{uanl}
		\end{subfigure}
		\hspace{15 mm}
		\begin{subfigure}{0.2\linewidth}
			\includegraphics[width=\linewidth]{fime}
		\end{subfigure}
	\end{figure}
	\vskip16mm
	\begin{tabular}{p{11cm}}
		\centering
		{\large Portafolio de Evidencias}
	\end{tabular}
	\vskip7mm
	{de}\\[7mm]
	{\large Nestor David Rodríguez Regalado}\\[3mm]
	{1460647}\\[7 mm]
	{curso de Simulación Computacional de Nanomateriales,}\\[3mm]
	{con el profesor Virgilio González en colaboración satu elisa schaeffer.}\\[3mm]
	Semestre Enero 2022 - Junio 2022. \\ [5 mm]
	\begin{table}[ht]
\centering

\begin{tabular}{|c|c|c|c|c|c|c|c|c|c|c|c|c|c|c|} 
 \hline
 \multicolumn{12}{|c|}{Práctica} & \multirow{2}{*}{T} & \multirow{2}{*}{P} & \multirow{2}{*}{C} \\
 \cline{1-12}
 t1 & t2 & t3 & t4 & t5 & t6 & t7 & t8 & t9 & t10 & t11 & t12 & & & \\
 \hline
 $5$ & $5$ & $5$ & $8$ & $6$ & $4$ & $7$ & $8$ & $6$ & $5$ & $4$ & $5$ & \multicolumn{1}{|c|}{68} & $4+4+5+3+0+4=20$ & \multicolumn{1}{|c|}{88}\\
 \hline
\end{tabular}
\label{Cuadro3}
\end{table}

	\url{https://github.com/NestorZeus/SIMULACION-COMPUTACIONAL-DE-NANOMATERIALES}
	\vfill
\end{center}
\end{scshape}

\end{document}